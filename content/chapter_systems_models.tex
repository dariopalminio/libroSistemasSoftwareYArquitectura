%% Chapter: Modelado de Sistemas

\chapter{Modelado de Sistemas}

Texto del Modelado de Sistemas

\section{Modelado}
\section{Herramientas conceptuales de modelado}

\subsection{Herramientas conceptuales de Diagramado}
\begin{table}[h]
\centering
\caption{Herramientas conceptuales de Diagramado}
\label{Herramientas-conceptuales-de-Diagramado}
\begin{tabular}{|l|l|}
\hline
Acrónimo & Significado en inglés                         \\ \hline
ADL      & Architecture Description Language             \\ \hline
BPMN     & Business Process Modeling Notation            \\ \hline
CD       & Conceptual Diagram or ConceptDraw             \\ \hline
CLD      & Causal Loop Diagram                           \\ \hline
ERD      & Entity Relationship Diagram                   \\ \hline
FC       & Flow Charts (for control flow)                \\ \hline
DFD      & Data Flow Diagram                             \\ \hline
MMD      & Map Mind Diagram                              \\ \hline
SC       & Structure Chart                               \\ \hline
SFD      & Stock and Flow Diagrams                       \\ \hline
SSADM    & Structured Systems Analysis and Design Method \\ \hline
UML      & Unified Modeling Language                     \\ \hline
\end{tabular}
\end{table}

\subsubsection{ADL}
\subsubsection{BPMN}
\subsubsection{CD}
\subsubsection{CLD}
\subsubsection{ERD}
\subsubsection{FC}
El Diagrama de Flujo (Flow Charts) es usado para diagramar flujos de control.

\subsubsection{DFD}
El Diagrama de Flujo de Datos o DFD fue introducido y popularizado en 1970 para el análisis y diseño estructurado \cite{Gane-Sarson-1979} para diagramar procesos de sistemas. Un DFD muestran el flujo de datos dentro del sistema desde entidades externas, mostrando como los datos fluyen entre diferentes procesos y qué almacenes intervienen para que los datos sean guardados y recuperados \cite{Scott-Ambler-2004}. 
Para este tipo de diagramas hay dos estándares de notaciones: la notación DeMarco-Yourdon (DeMarco, Yourdon y Constantine 1979 \cite{Dixit-2007}) y la notación Gane-Sarson.
%ver http://www.agilemodeling.com/artifacts/dataFlowDiagram.htm

En este tipo de diagramas hay solo cuatro elementos \cite{Dixit-2007}:
\begin{itemize}
\item Entidades externas: son fuentes esternas. En ambas notaciones se representa con cuadrados.
\item Procesos: son los subsistemas o procesos del sistema. En notación Gane-Sarson se representa con rectángulos redondeados y según DeMarco-Yourdon con círculos o elipses.  
\item Flujo de datos: flujo de datos electrónicos o físicos. En ambas notaciones se representa con flechas.
\item Almacenes: almacenes de información representadas. En notación Gane-Sarson se representa con rectángulos abiertos y según DeMarco-Yourdon con rectangulos sin líneas laterales.
\end{itemize}

\begin{figure}[h]
  \includegraphics{DFDGaneSarsonNotation}
  \caption{DFD Notación Gane-Sarson}
  \centering
  \label{fig:DFDGaneSarsonNotation} %\ref{fig:DFDGaneSarsonNotation}
\end{figure}

\begin{figure}[h]
  \includegraphics{DFDDeMarcoYourdonNotation}
  \caption{DFD Notación DeMarcoYourdon}
  \centering
  \label{fig:DFDDeMarcoYourdonNotation} %\ref{fig:DFDDeMarcoYourdonNotation}
\end{figure}


\subsubsection{MMD}
\subsubsection{SC}
\subsubsection{SFD}
\subsubsection{SSADM}
\subsubsection{UML}

\section{Modelos de un sistema general}
\section{Sistema general de modelado}
\section{Modelo de sistemas generales básicos}
\subsection{Sistemas termodinámicos}
\subsection{Sistema aislado en equilibrio termodinámico}
\subsection{Sistema aislado tendiente al equilibrio}
\subsection{Sistema abierto en equilibrio estacionario}
\subsection{Sistema Sistemas en Equilibrio Dinámico}
\subsection{Sistemas Sostenibles y Sostenibilidad}
\subsection{Sistema Mínimo de Vida}
\subsection{Sistema Agente}
\subsection{Sistema Autopoiético Mínimo}
\section{Modelos de Sistemas Cibernético}

\subsection{Sistema Cibernético General}
\subsubsection{Ejemplo 1: Sistema regulador de tanque de agua}
\subsubsection{Ejemplo 2: Sistema regulador de Watt}
\subsubsection{Sistema Cibernético Auto-Aprendiente}
\subsubsection{Sistema Cibernético Auto-Organizado}
\subsubsection{Sistema Cibernético Auto-Aprendiz y Auto-Organizado}

\subsection{Modelos de organizaciones}
\subsubsection{Viable System Model}
