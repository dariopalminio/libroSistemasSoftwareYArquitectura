%% Chapter: Modelado de Sistemas

\chapter{Modelado de Sistemas}

Texto del Modelado de Sistemas

\section{Modelado}
\section{Herramientas conceptuales de modelado}

\begin{table}[h]
\centering
\caption{Herramientas conceptuales de Diagramado}
\label{my-label}
\begin{tabular}{|l|l|}
\hline
Acrónimo & Detalle                                       \\ \hline
ADL      & Architecture Description Language             \\ \hline
BPMN     & Business Process Modeling Notation            \\ \hline
CD       & Conceptual Diagram                            \\ \hline
CLD      & Causal Loop Diagram                           \\ \hline
DFD      & Data Flow Diagram                             \\ \hline
MMD      & Map Mind Diagram                              \\ \hline
SC       & Structure Chart                               \\ \hline
SFD      & Stock and Flow Diagrams                       \\ \hline
SSADM    & Structured Systems Analysis and Design Method \\ \hline
UML      & Unified Modeling Language                     \\ \hline
\end{tabular}
\end{table}

\section{Modelos de un sistema general}
\section{Sistema general de modelado}
\section{Modelo de sistemas generales básicos}
\subsection{Sistemas termodinámicos}
\subsection{Sistema aislado en equilibrio termodinámico}
\subsection{Sistema aislado tendiente al equilibrio}
\subsection{Sistema abierto en equilibrio estacionario}
\subsection{Sistema Sistemas en Equilibrio Dinámico}
\subsection{Sistemas Sostenibles y Sostenibilidad}
\subsection{Sistema Mínimo de Vida}
\subsection{Sistema Agente}
\subsection{Sistema Autopoiético Mínimo}
\section{Modelos de Sistemas Cibernético}

\subsection{Sistema Cibernético General}
\subsubsection{Ejemplo 1: Sistema regulador de tanque de agua}
\subsubsection{Ejemplo 2: Sistema regulador de Watt}
\subsubsection{Sistema Cibernético Auto-Aprendiente}
\subsubsection{Sistema Cibernético Auto-Organizado}
\subsubsection{Sistema Cibernético Auto-Aprendiz y Auto-Organizado}

\subsection{Modelos de organizaciones}
\subsubsection{Viable System Model}
